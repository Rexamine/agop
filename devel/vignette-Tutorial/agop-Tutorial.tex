\documentclass[11pt]{article}\usepackage{graphicx, color}
%% maxwidth is the original width if it is less than linewidth
%% otherwise use linewidth (to make sure the graphics do not exceed the margin)
\makeatletter
\def\maxwidth{ %
  \ifdim\Gin@nat@width>\linewidth
    \linewidth
  \else
    \Gin@nat@width
  \fi
}
\makeatother

\IfFileExists{upquote.sty}{\usepackage{upquote}}{}
\definecolor{fgcolor}{rgb}{0.2, 0.2, 0.2}
\newcommand{\hlnumber}[1]{\textcolor[rgb]{0,0,0}{#1}}%
\newcommand{\hlfunctioncall}[1]{\textcolor[rgb]{0.501960784313725,0,0.329411764705882}{\textbf{#1}}}%
\newcommand{\hlstring}[1]{\textcolor[rgb]{0.6,0.6,1}{#1}}%
\newcommand{\hlkeyword}[1]{\textcolor[rgb]{0,0,0}{\textbf{#1}}}%
\newcommand{\hlargument}[1]{\textcolor[rgb]{0.690196078431373,0.250980392156863,0.0196078431372549}{#1}}%
\newcommand{\hlcomment}[1]{\textcolor[rgb]{0.180392156862745,0.6,0.341176470588235}{#1}}%
\newcommand{\hlroxygencomment}[1]{\textcolor[rgb]{0.43921568627451,0.47843137254902,0.701960784313725}{#1}}%
\newcommand{\hlformalargs}[1]{\textcolor[rgb]{0.690196078431373,0.250980392156863,0.0196078431372549}{#1}}%
\newcommand{\hleqformalargs}[1]{\textcolor[rgb]{0.690196078431373,0.250980392156863,0.0196078431372549}{#1}}%
\newcommand{\hlassignement}[1]{\textcolor[rgb]{0,0,0}{\textbf{#1}}}%
\newcommand{\hlpackage}[1]{\textcolor[rgb]{0.588235294117647,0.709803921568627,0.145098039215686}{#1}}%
\newcommand{\hlslot}[1]{\textit{#1}}%
\newcommand{\hlsymbol}[1]{\textcolor[rgb]{0,0,0}{#1}}%
\newcommand{\hlprompt}[1]{\textcolor[rgb]{0.2,0.2,0.2}{#1}}%

\usepackage{framed}
\makeatletter
\newenvironment{kframe}{%
 \def\at@end@of@kframe{}%
 \ifinner\ifhmode%
  \def\at@end@of@kframe{\end{minipage}}%
  \begin{minipage}{\columnwidth}%
 \fi\fi%
 \def\FrameCommand##1{\hskip\@totalleftmargin \hskip-\fboxsep
 \colorbox{shadecolor}{##1}\hskip-\fboxsep
     % There is no \\@totalrightmargin, so:
     \hskip-\linewidth \hskip-\@totalleftmargin \hskip\columnwidth}%
 \MakeFramed {\advance\hsize-\width
   \@totalleftmargin\z@ \linewidth\hsize
   \@setminipage}}%
 {\par\unskip\endMakeFramed%
 \at@end@of@kframe}
\makeatother

\definecolor{shadecolor}{rgb}{.97, .97, .97}
\definecolor{messagecolor}{rgb}{0, 0, 0}
\definecolor{warningcolor}{rgb}{1, 0, 1}
\definecolor{errorcolor}{rgb}{1, 0, 0}
\newenvironment{knitrout}{}{} % an empty environment to be redefined in TeX

\usepackage{alltt}
%\VignetteIndexEntry{A Guide to the agop Package}
%\VignetteEngine{knitr}

%%%%%%%%%%%%%%%%%%%%%%%%%%%%%%%%%%%%%%%%%%%%%%%%%%%%%%%%%%%%%%%%%%%%%%
%%%%%%%%%%%%%%%%%%%%%%%%%%%%%%%%%%%%%%%%%%%%%%%%%%%%%%%%%%%%%%%%%%%%%%
%%%%%%%%%%%%%%%%%%%%%%%%%%%%%%%%%%%%%%%%%%%%%%%%%%%%%%%%%%%%%%%%%%%%%%


\usepackage[dvips,a4paper,left=2.5cm,right=2.5cm,foot=1.0cm,
   headheight=1.0cm,top=2.0cm,margin=2.5cm]{geometry}
\linespread{1.1}

\usepackage{fancyhdr}


\usepackage[T1]{fontenc}
\usepackage[utf8]{inputenc}
\usepackage[english]{babel}
\selectlanguage{english}
\usepackage{xspace}
\usepackage{lmodern}

\usepackage{amsmath,amssymb,amsfonts,amsthm}
\RequirePackage{graphicx,verbatim,longtable}
\usepackage{mdwlist}
\usepackage{multirow,multicol}
\usepackage[nottoc]{tocbibind}
\usepackage{rotating}

\newcommand{\email}[1]{\href{mailto:#1}{#1}}
\renewcommand{\emph}[1]{\textsl{#1}}

\newcommand{\package}[1]{\textsf{#1}\xspace}
\newcommand{\program}[1]{\textsf{#1}\xspace}
\newcommand{\os}[1]{\textsf{#1}\xspace}
\newcommand{\lang}[1]{\textsf{#1}\xspace}
\newcommand{\Cpp}{\lang{C++}}
\newcommand{\R}{\lang{R}}

\newcommand{\Rfunc}[1]{\texttt{\hlfunctioncall{#1}}}
\newcommand{\argument}[1]{\texttt{\hlargument{#1}}}
\newcommand{\str}[1]{\texttt{\hlstring{#1}}}
\newcommand{\key}[1]{{$\langle$\texttt{#1}$\rangle$}\xspace} % skrot klawiszowy

\setlength{\pdfpagewidth}{\paperwidth}
\setlength{\pdfpageheight}{\paperheight}

% \renewcommand{\emph}{\textit}%
\newcommand{\indicator}{\text{\bf 1}}%
\renewcommand{\Pr}{\mathrm{P}}%
% \renewcommand{\ln}{\mathrm{ln}\,}%
\newcommand{\ran}{\mathrm{ran}\,}%
\newcommand{\img}{\mathrm{img}\,}%
\newcommand{\supp}{\mathrm{supp}\,}%

\newcommand{\Identity}{\mathsf{id}}


\newcommand{\vect}[1]{{\mathbf{#1}}}
\newcommand{\func}[1]{{\mathsf{#1}}}

\newcommand{\Reals}{\mathbb{R}}
\newcommand{\RealsExt}{\bar{\mathbb{R}}}
\newcommand{\Naturals}{\mathbb{N}}
\newcommand{\Ival}{\mathbb{I}}
\newcommand{\IvalPow}[1]{\mathbb{I}^{#1}}
\newcommand{\AnyPow}{^{1,2,\dots}}
\newcommand{\IvalAnyPow}{\mathbb{I}\AnyPow}
\newcommand{\Axiom}[1]{\mathrm{(#1)}}
\newcommand{\PropertyAny}[1]{\mathcal{#1}}
\newcommand{\Property}[1]{\mathcal{P}_\mathrm{(#1)}}

\usepackage{xcolor}
\definecolor{blue2}{rgb}{0,0.2,0.7}
\definecolor{red2}{rgb}{0.4,0.1,0.1}
\usepackage{colortbl}
\definecolor{navy}{rgb}{0,0.0,0.4}
\definecolor{navy2}{rgb}{0.4,0.1,0.3}
\definecolor{red2}{rgb}{0.6,0.1,0.2}
\definecolor{green2}{rgb}{0.1,0.4,0.2}

\newtheorem{theorem}{Theorem}%[section]
\newtheorem{lemma}[theorem]{Lemma}
\newtheorem{corollary}[theorem]{Corollary}
\newtheorem{proposition}[theorem]{Proposition}

\theoremstyle{remark}
\newtheorem{remark}[theorem]{Remark}

\theoremstyle{definition}
\newtheorem{definition}[theorem]{Definition}

\newtheorem{example}{Example}

\usepackage{hyperref}

\usepackage{caption}
\captionsetup{font=small,labelfont=bf,labelsep=period,justification=centering}
\addto\captionsenglish{\renewcommand{\figurename}{Fig.}}
\addto\captionsenglish{\renewcommand{\tablename}{Tab.}}


\setlength{\topsep}{1pt} % wpływa m.in. na odstęp dla verbatim
\tolerance=500
\predisplaypenalty=0
\clubpenalty=1000
\widowpenalty=1000



\newif\ifDevelopmentVersion

\DevelopmentVersiontrue




%%%%%%%%%%%%%%%%%%%%%%%%%%%%%%%%%%%%%%%%%%%%%%%%%%%%%%%%%%%%%%%%%%%%%%
%%%%%%%%%%%%%%%%%%%%%%%%%%%%%%%%%%%%%%%%%%%%%%%%%%%%%%%%%%%%%%%%%%%%%%
%%%%%%%%%%%%%%%%%%%%%%%%%%%%%%%%%%%%%%%%%%%%%%%%%%%%%%%%%%%%%%%%%%%%%%


\ifDevelopmentVersion
\pagestyle{fancy}
\fancyhead[L]{}
\fancyhead[R]{}
\fancyhead[C]{\footnotesize\sf This tutorial reflects the state of the most 
recent development version
of \package{agop} available on 
\href{https://github.com/Rexamine/agop}{\textit{GitHub}}.\newline
If you use the ``official'' CRAN release, some of the features may be unavailable.}
\fi

\begin{document}


\begin{center}
{\LARGE\sf A Guide to the \package{agop} {0.1-0} Package for \R}

{\large\sf\textit{Aggregation Operators in \R}}

\bigskip
{\large Marek Gagolewski${}^{1,2}$, Anna Cena${}^{1,2}$}

${}^{1}$ Systems Research Institute, Polish Academy of Sciences

ul. Newelska 6, 01-447 Warsaw, Poland

${}^{2}$ Rexamine, Email: \texttt{\{gagolews,cena\}@rexamine.com}

\href{http://www.rexamine.com/resources/agop/}%
{www.rexamine.com/resources/agop/}

\bigskip
\today


\medskip
\textit{The package, as well as this tutorial, is still in its early
days -- any suggestions are welcome!}
\end{center}





\bigskip\hrule\bigskip
\tableofcontents




% \definecolor{fgcolor}{gray}{0}
% \renewcommand{\hlnumber}[1]{\textcolor[gray]{0.2}{#1}}%
% \renewcommand{\hlfunctioncall}[1]{\textbf{#1}}%
% \renewcommand{\hlstring}[1]{\textcolor[gray]{0.2}{\textit{#1}}}%
% \renewcommand{\hlkeyword}[1]{\textbf{#1}}%
% \renewcommand{\hlargument}[1]{\textcolor[rgb]{0.2,0.2,0.2}{\textsl{#1}}}%
% \renewcommand{\hlcomment}[1]{\textcolor[gray]{0.5}{\it\textsf{#1}}}%
% \renewcommand{\hlroxygencomment}[1]{\textcolor[gray]{0.5}{\it\textsf{#1}}}%
% \renewcommand{\hlformalargs}[1]{\textcolor[rgb]{0.69,0.25,0.03}{#1}}%
% \renewcommand{\hleqformalargs}[1]{\textcolor[rgb]{0.69,0.25,0.03}{#1}}%
% \renewcommand{\hlassignement}[1]{\textcolor[gray]{0}{\textbf{#1}}}%
% \renewcommand{\hlpackage}[1]{\textcolor[rgb]{0.59,0.71,0.15}{#1}}%
% \renewcommand{\hlslot}[1]{\textit{#1}}%
% \renewcommand{\hlsymbol}[1]{\textcolor[cmyk]{0,0,0,1}{#1}}%
% \renewcommand{\hlprompt}[1]{\textcolor[cmyk]{0,0,0,0.5}{#1}}%

%%%%%%%%%%%%%%%%%%%%%%%%%%%%%%%%%%%%%%%%%%%%%%%%%%%%%%%%%%%%%%%%%%%%%
%%%%%%%%%%%%%%%%%%%%%%%%%%%%%%%%%%%%%%%%%%%%%%%%%%%%%%%%%%%%%%%%%%%%%
%%%%%%%%%%%%%%%%%%%%%%%%%%%%%%%%%%%%%%%%%%%%%%%%%%%%%%%%%%%%%%%%%%%%%




\section{Getting started}


intro.....
aggregation.... \cite{GrabischETAL2009:aggregationfunctions}



\R \cite{Rproject:home} is a free, open sourced software environment
for statistical computing and graphics, which
includes an implementation
of a very powerful and quite popular high-level language called \lang{S}.
It runs on all major operating systems, i.e.~\os{Windows},
\os{Linux}, and \os{MacOS X}.
To install \R and/or find some information on the \lang{S} language
please visit \R Project's Homepage at \href{http://www.R-project.org}{www.R-project.org}.
Perhaps you may also wish to install  \program{RStudio},
a convenient development environment for \R.
It is available at \href{http://rstudio.org/}{www.rsudio.org}.


\bigskip
\package{agop} is an Open Source (licensed under GNU LGPL 3)
package for \R$\ge$ 2.12 to which anyone can contribute.
It started as a fork of the \package{CITAN} (Citation
Analysis Toolpack) package for \R.

To install latest ``official'' release of the 
package available on \textit{CRAN} we type%
\ifDevelopmentVersion%
\footnote{You are viewing the \textbf{development} version of the tutorial.
Some of the features presented in this document may be missing
in the CRAN release. Please, upgrade to the \textbf{latest} development version from
\href{https://github.com/Rexamine/agop}{\textit{GitHub}}
if you need the new functionality.}\ignorespaces
\fi%
:

\begin{knitrout}\small
\definecolor{shadecolor}{rgb}{0.969, 0.969, 0.969}\color{fgcolor}\begin{kframe}
\begin{alltt}
\hlcomment{# install.packages('agop') # NOT YET AVAILABLE ON CRAN}
\end{alltt}
\end{kframe}
\end{knitrout}


\noindent
Alternatively, we may fetch its current development snapshot
from \href{https://github.com/Rexamine/agop}{\textit{GitHub}}:

\begin{knitrout}\small
\definecolor{shadecolor}{rgb}{0.969, 0.969, 0.969}\color{fgcolor}\begin{kframe}
\begin{alltt}
\hlfunctioncall{install.packages}(\hlstring{'devtools'})
\hlfunctioncall{library}(\hlstring{'devtools'})
\hlfunctioncall{install_github}(\hlstring{'agop'}, \hlstring{'Rexamine'})
\end{alltt}
\end{kframe}
\end{knitrout}




\bigskip
Each session with \package{agop} should be preceded by
a call to:

\begin{knitrout}\small
\definecolor{shadecolor}{rgb}{0.969, 0.969, 0.969}\color{fgcolor}\begin{kframe}
\begin{alltt}
\hlfunctioncall{library}(\hlstring{'agop'}) \hlcomment{# Load the package}
\end{alltt}
\end{kframe}
\end{knitrout}


\bigskip
To view the main page of the manual we type:

\begin{knitrout}\small
\definecolor{shadecolor}{rgb}{0.969, 0.969, 0.969}\color{fgcolor}\begin{kframe}
\begin{alltt}
\hlfunctioncall{library}(help=\hlstring{'agop'})
\end{alltt}
\end{kframe}
\end{knitrout}


\noindent
For more information please visit the package's homepage \cite{agopHome}.
In case of any problems, comments, or suggestions feel free to contact the authors.
Good luck!



%%%%%%%%%%%%%%%%%%%%%%%%%%%%%%%%%%%%%%%%%%%%%%%%%%%%%%%%%%%%%%%%%%%%%
%%%%%%%%%%%%%%%%%%%%%%%%%%%%%%%%%%%%%%%%%%%%%%%%%%%%%%%%%%%%%%%%%%%%%
%%%%%%%%%%%%%%%%%%%%%%%%%%%%%%%%%%%%%%%%%%%%%%%%%%%%%%%%%%%%%%%%%%%%%




\section{Theoretical Background}

Let us begin with some basic notation convention. 
From now on let $\Ival=[a,b]$, possibly with $a=-\infty$ or $b=\infty$
(in many practical situations we choose $\Ival=[0,1]$ or $\Ival=[0,\infty]$).
A set of all vectors with elements in $\Ival$ of arbitrary length 
is denoted by $\IvalAnyPow=\bigcup_{n=1}^\infty \IvalPow{n}$ .

For $\vect{x},\vect{y}\in\IvalPow{n}$
we write $\vect{x}\le\vect{y}$ if and only if for all $i$ it holds $x_i\le y_i$.
Moreover, all binary arithmetic operations on vectors $\vect{x},\vect{y}\in\IvalPow{n}$
are performed element-wise,
e.g. $\vect{x}+\vect{y}=(x_1+y_1,\dots,x_n+y_n)\in\IvalPow{n}$.
Similarly: $-$, $\cdot$, $/$, $\wedge$ (min), $\vee$ (max), etc.
Additionally, each function of one variable
$\func{f}:\Ival\to\Ival$ can be extended to the vector
space: we write $\func{f}(\vect{x})=(\func{f}(x_1), \dots, \func{f}(x_n))$.

Let $x_{(i)}$ denote the $i$th order statistic,
i.e.~the $i$th smallest value in $\vect{x}$.
Moreover, for convenience,
let $x_{\{i\}}=x_{|\vect{x}|-i+1}$ denote the $i$th greatest value in $\vect{x}$.

For any $k\in\mathbb{N}$ and $c\in\Ival$ , we set $(n\ast c) = (c,\dots,c)\in\IvalPow{n}$.



\subsection{Aggregation Operators and Their Basic Properties}

\begin{definition}
$\func{F}: \IvalAnyPow\to\Ival$ is called an \emph{(extended) aggregation operator}
(cf.~\cite{GrabischETAL2009:aggregationfunctions})
if it is at least \emph{nondecreasing} in each variable,
i.e.~for all $n$ and $\vect{x},\vect{y}\in\IvalPow{n}$
if $\vect{x}\le\vect{y}$, then $\func{F}(\vect{x})\le\func{F}(\vect{y})$.
\end{definition}

Note that each aggregation operator is a mapping into $\Ival$,
thus for all $n$ we have $\inf_{\vect{x}\in\IvalPow{n}} \func{F}(\vect{x}) \ge a$
and $\sup_{\vect{x}\in\IvalPow{n}} \func{F}(\vect{x}) \le b$.
By nondecreasingness, however, these conditions reduce to
$\func{F}(n\ast a)\ge a$ and $\func{F}(n\ast b)\le b$.

\begin{definition}
We call $\func{F}: \IvalAnyPow\to\Ival$ \emph{symmetric} if \[
(\forall n\in\mathbb{N})\ (\forall \vect{x},\vect{y}\in\IvalPow{n})\
\vect{x}\cong\vect{y}\Longrightarrow\func{F}(\vect{x})=\func{F}(\vect{y}),
 \] where  $\vect{x}\cong\vect{y}$ if and only if there exists a permutation
$\sigma$ of $[n]:=\{1,2,\dots,n\}$ such that
$\vect{x}=(y_{\sigma(1)},\dots,y_{\sigma(n)})$
\end{definition}
\medskip
It may be shown, see \cite{GrabischETAL2009:aggregationfunctions}, that 
$\func{F}:\IvalPow{n}\to\Ival$ is summetric if and only if there exist 
function $\func{G}:\IvalPow{n}\to\Ival$ such that 
$\func{F}(x_{1},\dots,x_{n})=\func{G}(x_{(1)},\dots,x_{(n)})$.
\bigskip
\begin{definition}
We call $\func{F}: \IvalAnyPow\to\Ival$ \emph{idempotent} if 
\[(\forall x\in\Ival)\ \func{F}(n\ast x)=x.\]
\end{definition}

\begin{definition}
We call $\func{F}: \IvalAnyPow\to\Ival$ \emph{additive} if \[
\func{F}(\vect{x}+\vect{y})=\func{F}(\vect{x})+\func{F}(\vect{y}),
\]
for all $\vect{x},\vect{y}\in\IvalPow{n}$ such that 
$\vect{x}+\vect{y}\in\IvalPow{n}$.
\end{definition}

\begin{definition}
We call $\func{F}$ \emph{minitive} if \[(\forall n)\ (\forall \vect{x},\vect{y}\in\IvalPow{n})\ 
\func{F}(\vect{x}\wedge
\vect{y})=\func{F}(\vect{x})\wedge\func{F}(\vect{y}).\]

\end{definition}

\begin{definition}
We call $\func{F}$ \emph{maxitive} if \[(\forall n)\ (\forall \vect{x},\vect{y}\in\IvalPow{n})\
\func{F}(\vect{x}\vee
\vect{y})=\func{F}(\vect{x})\vee\func{F}(\vect{y}).\] 
\end{definition}

\begin{definition}
We call $\func{F}$ \emph{symmetric modular} if \[(\forall n)\ (\forall \vect{x},\vect{y}\in\IvalPow{n})\ 
\func{F}(\vect{x}\stackrel{S}{\vee}\vect{y})
+\func{F}(\vect{x}\stackrel{S}{\wedge}\vect{y})=\func{F}(\vect{x})+\func{F}
(\vect{y}),\] 
where $\vect{x}\stackrel{S}{\vee  }\vect{y}=(x_{(1)}\vee
y_{(1)},\dots,x_{(n)}\vee   y_{(n)})$
and
$\vect{x}\stackrel{S}{\wedge}\vect{y}=(x_{(1)}\wedge y_{(1)},\dots,x_{(n)}\wedge
y_{(n)})$.
\end{definition}

\subsection{Impact Functions and The Producers Assessment Problem}

..........

Let $\Ival=[0,\infty]$ represent the set of values that some a~priori chosen
paper quality measure may take. These may of course be non-integers,
for example when we consider citations normalized with respect
to the number of papers' authors.


It is widely accepted, see e.g.~\cite{Woeginger2008:axiomatich,
Woeginger2008:axiomaticg,
Woeginger2008:symmetryaxiom,Rousseau2008:woegingerax,Quesada2009:monotonicityh,
Quesada2010:moreaxiomatics,GagolewskiGrzegorzewski2011:ijar,
Gagolewski2011:PhD,FranceschiniMaisano2011:structevalh},
that each aggregation operator
 $\func{J}:\IvalAnyPow\to\Ival$ to be applied in the impact assessment
 process  should at least be:
\begin{enumerate}
   \item[(a)] nondecreasing in each variable (additional
   citations received by a paper or an improvement
   of its quality measure does not result
   in a decrease of the authors' overall evaluation),
   \item[(b)] arity-monotonic (by publishing a new paper we
   never decrease the overall valuation of the entity),
   \item[(c)] symmetric (independent
of the order of elements' presentation, i.e.~we may always assume that
we aggregate vectors that are already sorted).
\end{enumerate}
Conditions (a) and (b) imply that each impact function
is able -- at least potentially -- to describe two ``dimensions''
of the author's output quality: (a) his/her ability to write
eagerly-cited or highly-valuated papers and (b) his/her overall productivity.

More formally, condition (a) holds if and only if for each
$n$ and $\vect{x},\vect{y}\in\IvalPow{n}$ such that $(\forall i)$
$x_i\le y_i$ we have $\func{J}(\vect{x})\le\func{J}(\vect{y})$.
On the other hand, axiom (b) is fulfilled iff
for any $\vect{x}\in\IvalAnyPow$ and $y\in\Ival$ it holds
$\func{J}(\vect{x})\le\func{J}(x_1,\dots,x_n,y)$.
Lastly, requirement (c) holds iff for all $n$ and $\vect{x}\in\IvalPow{n}$
we have $\func{J}(\vect{x})=\func{J}(x_{\{1\}},\dots,x_{\{n\}})$,
where $x_{\{i\}}$ denotes the $i$th largest value from $\vect{x}$,
i.e.~its $(n-i+1)$th order statistic.




%%%%%%%%%%%%%%%%%%%%%%%%%%%%%%%%%%%%%%%%%%%%%%%%%%%%%%%%%%%%%%%%%%%%%
%%%%%%%%%%%%%%%%%%%%%%%%%%%%%%%%%%%%%%%%%%%%%%%%%%%%%%%%%%%%%%%%%%%%%
%%%%%%%%%%%%%%%%%%%%%%%%%%%%%%%%%%%%%%%%%%%%%%%%%%%%%%%%%%%%%%%%%%%%%




\section{Predefined Classes of Aggregation Operators in \package{agop}}


\subsection{A Note on Representing Numeric Data and Applying Functions in \R}
Generally, in our implementation we most often deal with numeric vectors.
Recall how we create them in \R:

\begin{knitrout}\small
\definecolor{shadecolor}{rgb}{0.969, 0.969, 0.969}\color{fgcolor}\begin{kframe}
\begin{alltt}
(x1 <- \hlfunctioncall{c}(5, 2, 3, 1, 0, 0))
\end{alltt}
\begin{verbatim}
## [1] 5 2 3 1 0 0
\end{verbatim}
\begin{alltt}
\hlfunctioncall{class}(x1)
\end{alltt}
\begin{verbatim}
## [1] "numeric"
\end{verbatim}
\begin{alltt}
(x2 <- \hlfunctioncall{rep}(10, 3))
\end{alltt}
\begin{verbatim}
## [1] 10 10 10
\end{verbatim}
\begin{alltt}
(x3 <- 10:1) \hlcomment{# the same as \hlfunctioncall{seq}(10, 1)}
\end{alltt}
\begin{verbatim}
##  [1] 10  9  8  7  6  5  4  3  2  1
\end{verbatim}
\begin{alltt}
(x4 <- \hlfunctioncall{seq}(1, 5, length.out=6))
\end{alltt}
\begin{verbatim}
## [1] 1.0 1.8 2.6 3.4 4.2 5.0
\end{verbatim}
\begin{alltt}
(x5 <- \hlfunctioncall{seq}(1, 5, by=1.25))
\end{alltt}
\begin{verbatim}
## [1] 1.00 2.25 3.50 4.75
\end{verbatim}
\end{kframe}
\end{knitrout}


Sometimes we will store the vectors of the same length
in a matrix (column-/row-wise.... col/rownames....) \Rfunc{apply()}....

\begin{knitrout}\small
\definecolor{shadecolor}{rgb}{0.969, 0.969, 0.969}\color{fgcolor}\begin{kframe}
\begin{alltt}
expertopinions <- \hlfunctioncall{matrix}(\hlfunctioncall{c}(
      6,7,2,3,1, \hlcomment{# this will be the first COLUMN}
      8,3,2,1,9, \hlcomment{# 2nd}
      4,2,4,1,6  \hlcomment{# 3rd}
   ),
   ncol=3,
   dimnames=\hlfunctioncall{list}(NULL, \hlfunctioncall{c}(\hlstring{"A"}, \hlstring{"B"}, \hlstring{"C"})) # only column names set
)
\hlfunctioncall{class}(expertopinions)
\end{alltt}
\begin{verbatim}
## [1] "matrix"
\end{verbatim}
\begin{alltt}
\hlfunctioncall{print}(expertopinions)   \hlcomment{# or \hlfunctioncall{print}(authors)}
\end{alltt}
\begin{verbatim}
##      A B C
## [1,] 6 8 4
## [2,] 7 3 2
## [3,] 2 2 4
## [4,] 3 1 1
## [5,] 1 9 6
\end{verbatim}
\begin{alltt}
\hlfunctioncall{apply}(expertopinions, 2, mean) \hlcomment{# on each COLUMN apply the \hlfunctioncall{mean}() function}
\end{alltt}
\begin{verbatim}
##   A   B   C 
## 3.8 4.6 3.4
\end{verbatim}
\end{kframe}
\end{knitrout}


...or in a list, especially when they are not of the same length....
\Rfunc{lapply()}.... \Rfunc{sapply()}..... possibly named elements...

\begin{knitrout}\small
\definecolor{shadecolor}{rgb}{0.969, 0.969, 0.969}\color{fgcolor}\begin{kframe}
\begin{alltt}
authors <- \hlfunctioncall{list}(
   \hlstring{"John S."} = \hlfunctioncall{c}(7,6,2,1,0),
   \hlstring{"Kate F."} = \hlfunctioncall{c}(9,8,7,6,4,1,1,0)
)
\hlfunctioncall{class}(authors)
\end{alltt}
\begin{verbatim}
## [1] "list"
\end{verbatim}
\begin{alltt}
\hlfunctioncall{str}(authors)   \hlcomment{# or \hlfunctioncall{print}(authors)}
\end{alltt}
\begin{verbatim}
## List of 2
##  $ John S.: num [1:5] 7 6 2 1 0
##  $ Kate F.: num [1:8] 9 8 7 6 4 1 1 0
\end{verbatim}
\begin{alltt}
\hlfunctioncall{index_h}(authors[[1]]) \hlcomment{# the h-\hlfunctioncall{index} (see below) for 1st author}
\end{alltt}
\begin{verbatim}
## [1] 2
\end{verbatim}
\begin{alltt}
\hlfunctioncall{sapply}(authors, index_h) \hlcomment{# calculate the h-index for all vectors in a list}
\end{alltt}
\begin{verbatim}
## John S. Kate F. 
##       2       4
\end{verbatim}
\begin{alltt}
\hlfunctioncall{index_h}(authors) \hlcomment{# \hlfunctioncall{index_h}() expects an numeric vector on input}
\end{alltt}


{\ttfamily\noindent\bfseries\color{errorcolor}{\#\# Error: argument `x` should be a numeric vector (or an object coercible to)}}\end{kframe}
\end{knitrout}




\subsection{...}

Ja to jeszcze zmienie, narazie wrzucam definicje ale potem je poukladam zeby bylo tak spojnie 
i co jest czyms, mozliwe ze czesc zniknie.

% weighted mean ,OWA (L-statistics), qL
% 
% WMin, OWMin, qI
% 
% WMax, OWMax (S-statistics), qS 
% 
% OM3


\begin{definition}
The \emph{weighted arithmetic mean} $\func{WAM}: \IvalPow{n}\to\Ival$ associated 
with the weight vector $\vect{w}=c(w_{1},\dots,w_{n})\in[0,1]^{n}$ such that 
$\sum_{i=1}^{n}w_{i}=1$ is defined as 
\[
\func{WAM}(\vect{x})=\sum_{i=1}^{n}w_{i}x_{i}
\] for any $\vect{x}\in\IvalPow{n}$.
\end{definition}

\begin{definition}
The \emph{ordered weighted aaveraging function} $\func{OWA}: \IvalPow{n}\to\Ival$ associated 
with the weight vector $\vect{w}=c(w_{1},\dots,w_{n})\in[0,1]^{n}$ such that 
$\sum_{i=1}^{n}w_{i}=1$ is defined as 
\[
\func{OWA}(\vect{x})=\sum_{i=1}^{n}w_{i}x_{(i)}
\] for any $\vect{x}\in\IvalPow{n}$.
\end{definition}

 (L-statistics)
\begin{definition}
The \emph{triangle of coefficients} is a sequence $\triangle=(c_{i,n}\in\Ival: i\in [n], n\in\Naturals)$.
\end{definition}

\begin{definition}
The \emph{triangle of functions} is a sequence $\triangle=(f_{i,n}\in\IvalPow{\Ival}: i\in [n], n\in\Naturals)$.
\end{definition}

\begin{definition}
The \emph{L-statistic} for a given triangle of coefficients $\triangle=(c_{i,n})_{ i\in [n],n\in\Naturals}$
is a function $\func{L}_{\triangle}:\IvalPow{n}\to\Ival$ such that 
\[
\func{L}_{\triangle}(\vect{x})=\sum_{i=1}^{n}c_{i,n}x_{(n-i+1)},
\] for any $\vect{x}\in\IvalPow{n}$.
\end{definition}

\begin{definition}
The \emph{quasi-L-statistic} for a given triangle of functions $\triangle=(f_{i,n})_{ i\in [n],n\in\Naturals}$
is a function $\func{qL}_{\triangle}:\IvalPow{n}\to\Ival$ such that 
\[
\func{qL}_{\triangle}(\vect{x})=\sum_{i=1}^{n}f_{i,n}(x_{(n-i+1)}),
\] for any $\vect{x}\in\IvalPow{n}$.
\end{definition}

WMax, OWMax 

\begin{definition}
The \emph{S-statistic} for a given triangle of coefficients $\triangle=(c_{i,n})_{ i\in [n],n\in\Naturals}$
is a function $\func{S}_{\triangle}:\IvalPow{n}\to\Ival$ such that 
\[
\func{S}_{\triangle}(\vect{x})=\bigvee_{i=1}^{n}c_{i,n}\wedge x_{(n-i+1)},
\] for any $\vect{x}\in\IvalPow{n}$.
\end{definition}

\begin{definition}
The \emph{quasi-S-statistic} for a given triangle of functions $\triangle=(f_{i,n})_{ i\in [n],n\in\Naturals}$
is a function $\func{qS}_{\triangle}:\IvalPow{n}\to\Ival$ such that 
\[
\func{qS}_{\triangle}(\vect{x})=\bigvee_{i=1}^{n}f_{i,n}(x_{(n-i+1)}),
\] for any $\vect{x}\in\IvalPow{n}$.
\end{definition}


WMin, OWMin

\begin{definition}
The \emph{I-statistic} for a given triangle of coefficients $\triangle=(c_{i,n})_{ i\in [n],n\in\Naturals}$
is a function $\func{I}_{\triangle}:\IvalPow{n}\to\Ival$ such that 
\[
\func{I}_{\triangle}(\vect{x})=\bigwedge_{i=1}^{n}c_{i,n}\vee x_{(n-i+1)},
\] for any $\vect{x}\in\IvalPow{n}$.
\end{definition}

\begin{definition}
The \emph{quasi-I-statistic} for a given triangle of functions $\triangle=(f_{i,n})_{ i\in [n],n\in\Naturals}$
is a function $\func{qI}_{\triangle}:\IvalPow{n}\to\Ival$ such that 
\[
\func{qI}_{\triangle}(\vect{x})=\bigwedge_{i=1}^{n}f_{i,n}(x_{(n-i+1)}),
\] for any $\vect{x}\in\IvalPow{n}$.
\end{definition}

Let us introduce now the class of symmetric maxitive, minitive and modular aggregation operators.

\begin{definition}
A sequence of nondecreasing functions $\vect{w}=(\func{w}_1,\func{w}_2,\dots)$,
$\func{w}_i :\Ival\to\Ival$,
 and a triangle of coefficients
 $\triangle=(c_{i,n})_{i\in[n],n\in\Naturals}$, $c_{i,n}\in\Ival$
 such that $(\forall n)$ $c_{1,n}\le c_{2,n}\le \dots \le c_{n,n}$,
 $0\le\func{w}_n(0)\le c_{1,n}$, and
 $\func{w}_n(b)=c_{n,n}$,
 generates a nondecreasing \emph{OM3 operator} $\func{M}_{\triangle,\vect{w}}:\IvalPow{n}\to\Ival$
 such that for $\vect{x}\in\IvalPow{n}$ we have:
\begin{eqnarray*}
\func{M}_{\triangle,\vect{w}}(\mathbf{x})&=&\bigvee_{i=1}^n
\func{w}_n(x_{(n-i+1)})\wedge c_{i,n} =  \bigwedge_{i=1}^n (\func{w}_n(x_{(n-i+1)}) \vee c_{i-1,n}) \wedge c_{n,n}\\
&=& \sum_{i=1}^n \left(\left(\func{w}_n(x_{(n-i+1)})\vee c_{i-1,n}\right) \wedge
c_{i,n} - c_{i-1,n}\right).
\end{eqnarray*}
\end{definition}

We see that the above contains i.a.~all
order statistics (whenever $\func{w}_n(x)=x$,
and $c_{i,n}=0$, $c_{j,n}=b$ for $i< k$, $j\ge k$, and some $k$),
OWMax operators (for $\func{w}_n(x)=x$),
and the famous Hirsch $h$-index ($\func{w}_n(x)=\lfloor x\rfloor$, $c_{i,n}=i$).


\subsection{Bibliometric Impact Indices}

Below we assume that $\Ival=[0,\infty]$.
% All the functions are symmetric. The input vectors are of course
% being sorted if necessary during calculations.


\paragraph{The $h$-index.}
Given a sequence $\vect{x}=(x_1,\dots,x_n)\in\IvalAnyPow$,
the \emph{Hirsch index} \cite{Hirsch2005:hindex} of $\vect{x}$ is defined as
$\func{H}(x)=\max\{i=1,\dots,n: x_{\{i\}} \ge i\}$
if $n \ge 1$ and $x_{\{1\}} \ge 1$, or $\func{H}(x)=0$ otherwise.
It may be shown that the $h$-index is an zero-insensitive
OM3 aggeration operator,
see \cite{Gagolewski2013:om3}, with:
\[
   \func{H}(\vect{x}) = \bigvee_{i=1,\dots,n}^n i\wedge \lfloor x_{\{i\}}\rfloor.
\]
Interpretation: ``an author has $h$-index of $H$ if $H$ of his/her
$n$ most cited papers have at least $H$ citations each, and the other $n-H$
papers are cited no more that $H$ times each''.
The $h$-index may also be expressed as a Sugeno 
integral \cite{Sugeno1974:PhD}
w.r.t.~to a counting measure, cf.~\cite{TorraNarukawa2008:h2fuzzyintegrals}.

\package{agop} implementation: \Rfunc{index\_h()}.

\begin{knitrout}\small
\definecolor{shadecolor}{rgb}{0.969, 0.969, 0.969}\color{fgcolor}\begin{kframe}
\begin{alltt}
\hlfunctioncall{index_h}(\hlfunctioncall{c}(6,5,4,2,1,0,0,0,0,0,0))
\end{alltt}
\begin{verbatim}
## [1] 3
\end{verbatim}
\begin{alltt}
\hlfunctioncall{index_h}(\hlfunctioncall{c}(-1,3,4,2)) \hlcomment{# only for x>=0}
\end{alltt}


{\ttfamily\noindent\bfseries\color{errorcolor}{\#\# Error: all elements in `x` should be in [0,Inf]}}\end{kframe}
\end{knitrout}


This is a zero-insensitive impact function.

We have $\func{H}(\vect{x})\le \min\{n, x_{1}\}$.

\paragraph{The $g$-index.}
Egghe's $g$-index \cite{Egghe2006:g}:
$\func{G}(\vect{x})=\max\{g=1,\dots,n: \sum_{i=1}^g x_{\{g\}}\ge g^2\}$,
available in \package{agop} as \Rfunc{index\_g()}.
We have $\func{G}(\vect{x})\ge \func{H}(\vect{x})$
with $\func{G}(n\ast n)=\func{H}(n\ast n)=n$


Note that this aggregation operator is not zero-insensitive,
for example $\func{G}(9,0)=2$ and $\func{G}(9,0,0)=3$.
Thus, we also provide the \Rfunc{index\_g\_zi()} function,
which treats $\vect{x}$ as it would be padded with $0$s.


\begin{knitrout}\small
\definecolor{shadecolor}{rgb}{0.969, 0.969, 0.969}\color{fgcolor}\begin{kframe}
\begin{alltt}
\hlfunctioncall{index_g}(9)
\end{alltt}
\begin{verbatim}
## [1] 1
\end{verbatim}
\begin{alltt}
\hlfunctioncall{index_g}(\hlfunctioncall{c}(9,0,0))
\end{alltt}
\begin{verbatim}
## [1] 3
\end{verbatim}
\begin{alltt}
\hlfunctioncall{index_g_zi}(9)
\end{alltt}
\begin{verbatim}
## [1] 3
\end{verbatim}
\end{kframe}
\end{knitrout}


The index is interesting from the computational point of view --
it may be calculated on the nondecreasing vector of cumulative sums,
\texttt{\Rfunc{cumsum}(\Rfunc{sort}(x, \argument{decreasing=}TRUE))},
however, it cannot be expressed as a symmetric maxitive aggregation operator.

Interestingly, it might be shown that if $\vect{x}$ is sorted
nondecreasingly, then:
\[
\func{G}(\vect{x}) = \func{H}(\vect{x})(0\vee \mathtt{cummin}(\mathtt{cumsum}(x)-(1:n)^2+(1:n))),
\]
where $1:n = (1, 2, 3, \dots, n)$.

\paragraph{The $w$-index.}
 The $w$-index \cite{Woeginger2008:axiomatich}:
\begin{equation}\label{Eq:IndexW}
\func{W}(\vect{x}) = \max\left\{w=0,1,2,\ldots: {x}_{\{i\}} \ge w-i+1, i=1,\dots,w\right\}.
\end{equation}

\package{agop} implementation: \Rfunc{index\_w()}.

This is a zero-insensitive impact function.

We have $\func{H}(\vect{x})\le\func{W}(\vect{x})\le 2\func{H}(\vect{x})$,
$\func{W}(\vect{x})\le \min\{n, x_{1}\}$.




Interestingly, it might be shown that if $\vect{x}$ is sorted
nondecreasingly, then:
\[
\func{W}(\vect{x}) = \func{H}(\vect{x})(\mathtt{cummin}(\vect{x}+(1:n)-1)).
\]

\paragraph{The $r_p$-indices.}\cite{GagolewskiGrzegorzewski2009:geometricapproach}
for integer vectors we have $r_1=\func{W}$ and $r_\infty=\func{H}$

This is a zero-insensitive impact function.





\paragraph{The $\mathrm{MAXPROD}$-index.}
The MaxProd-index \cite{Kosmulski2007:maxprod}:
\begin{equation}\label{Eq:IndexMaxProd}
   \func{MP}(\vect{x}) = \max\left\{i\cdot{{x}}_{\{i\}}: i=1,2,\ldots\right\}.
\end{equation}
This index is a particular case of a projected $l_\infty$-index,
see \cite{GagolewskiGrzegorzewski2009:geometricapproach}.

Shilkret integral \cite{Shilkret1971:maxitivemeasure}

This is a zero-insensitive impact function.

\package{agop} implementation: \Rfunc{index\_maxprod()}.




\paragraph{The $l_p$-indices.}\cite{GagolewskiGrzegorzewski2009:geometricapproach}


\paragraph{Simple transformations of the $h$-index.}
For example, The $h(2)$-index \cite{Kosmulski2006:h2}:
\begin{equation}\label{Eq:IndexH2}
\func{H2}(\vect{x}) = \max\left\{h=0,1,2,\ldots: {x}_{h} \ge h^2\right\}. %= \left\lfloor \bigvee_{i=1,2,\dots} i\wedge x_{i}\right\rfloor
\end{equation}
Note that the $h(2)$-index is one of the many examples of very
simple, direct modifications of the $h$-index.
Many authors considered other settings than ``$h^2$'' on the right
side of \eqref{Eq:IndexH2}, e.g. ``$2h$'', ``$\alpha h$'' for some
$\alpha > 0$, or ``$h^\beta$'', $\beta\ge 1$, cf.~\cite{AlonsoETAL2009:hreview}.

It may easily be shown that these reduce to the $h$-index for
properly transformed input vectors.....

%%%%%%%%%%%%%%%%%%%%%%%%%%%%%%%%%%%%%%%%%%%%%%%%%%%%%%%%%%%%%%%%%%%%%
%%%%%%%%%%%%%%%%%%%%%%%%%%%%%%%%%%%%%%%%%%%%%%%%%%%%%%%%%%%%%%%%%%%%%
%%%%%%%%%%%%%%%%%%%%%%%%%%%%%%%%%%%%%%%%%%%%%%%%%%%%%%%%%%%%%%%%%%%%%


\section{Visualization}


\subsection{Depicting producers}

The \Rfunc{plot\_producer()} function may be used to draw
a graphical representation of a given numeric vector,
i.e.~what is sometimes called a citation function in scientometrics.

A given vector $\mathbf{x}=(x_1,\dots,x_n)$ can be represented by a
step function defined for $0\le y<n$ and given by:
\[
   \pi(y)=x_{(n-\lfloor y+1\rfloor+1)}.
\]
This function may be obtained by setting \texttt{type == 'right.continuous'}
argument in \Rfunc{plot\_pro\-du\-cer()}.
Recall that $x_{(i)}$ denotes $i$-th smallest value in $\vect{x}$.

On the other hand, for \texttt{type == 'left.continuous'}
(the default), we get
\[
\pi(y)=x_{(n-\lfloor y\rfloor+1)}
\]
for $0< y\le n$.

Moreover, this function may depict the curve joining the sequence
of points $(0, x_{(n)}), (1, x_{(n)}),\allowbreak (1, x_{(n-1)}), (2, x_{(n-1)}),
\dots, (n, x_{(1)})$.


\medskip
The \Rfunc{plot\_producer()} function behaves much like
the well-known \R's \Rfunc{plot.default()} and allows for passing
all its graphical parameters.

For example, let us depict the state of two given producers,
$\vect{x}^{(1)}$ and $\vect{x}^{(2)}$.

\begin{knitrout}\small
\definecolor{shadecolor}{rgb}{0.969, 0.969, 0.969}\color{fgcolor}\begin{kframe}
\begin{alltt}
x1 <- \hlfunctioncall{c}(5, 4, 2, 2, 1)
x2 <- \hlfunctioncall{c}(3, 3, 1, 0, 0, 0, 0)
\hlfunctioncall{plot_producer}(x1, extend=TRUE)
\hlfunctioncall{plot_producer}(x2, add=TRUE, col=2, pch=2, extend=TRUE)
\hlfunctioncall{legend}(\hlstring{'topright'}, \hlfunctioncall{c}(\hlstring{'x1'}, \hlstring{'x2'}), col=\hlfunctioncall{c}(1, 2), lty=1, pch=\hlfunctioncall{c}(1, 2))
\end{alltt}
\end{kframe}
\end{knitrout}


\begin{center}
\scalebox{0.83}{
\begin{knitrout}\small
\definecolor{shadecolor}{rgb}{0.969, 0.969, 0.969}\color{fgcolor}

{\centering \includegraphics[width=4.5in]{figures-knitr/vis1b} 

}



\end{knitrout}

}
\end{center}



%%%%%%%%%%%%%%%%%%%%%%%%%%%%%%%%%%%%%%%%%%%%%%%%%%%%%%%%%%%%%%%%%%%%%
%%%%%%%%%%%%%%%%%%%%%%%%%%%%%%%%%%%%%%%%%%%%%%%%%%%%%%%%%%%%%%%%%%%%%
%%%%%%%%%%%%%%%%%%%%%%%%%%%%%%%%%%%%%%%%%%%%%%%%%%%%%%%%%%%%%%%%%%%%%


\section{Pre-orders}

..............


Let us consider the following relation on $\IvalAnyPow$.
For any $\vect{x}\in\IvalPow{n}$ and $\vect{y}\in\IvalPow{m}$ we write
% \begin{equation}\label{Eq:partialorderingtrianglelefteq}
$\vect{x}\trianglelefteq\vect{y}$ if and only if
$n\le m\text{ and }x_{\{i\}}\le y_{\{i\}}$ for all $i\in\min\{n,m\}$.
Of course, $\trianglelefteq$ is a pre-order -- it
would have been a partial order, if we had defined it on the set
of \textit{sorted} vectors.



In other words, we  say that an author $X$
 is (weakly) dominated %(in sense of $\trianglelefteq$)
by an author $Y$, if $X$ has no more papers than $Y$ and each
the $i$th most cited paper of $X$ has no more citations than
the $i$th most cited paper of $Y$. Not that the $m-n$ least cited
$Y$'s papers are not taken into account here.
Most importantly, however,
there exist pairs of vectors
that are \textit{incomparable} with respect to $\trianglelefteq$ (see
the illustration below).

This pre-order in \package{agop} as \Rfunc{pord\_weakdom()}.

\begin{knitrout}\small
\definecolor{shadecolor}{rgb}{0.969, 0.969, 0.969}\color{fgcolor}\begin{kframe}
\begin{alltt}
\hlfunctioncall{c}(\hlfunctioncall{pord_weakdom}(5:1, 10:1), \hlfunctioncall{pord_weakdom}(10:1, 5:1)) \hlcomment{# 5:1 <= 10:1}
\end{alltt}
\begin{verbatim}
## [1]  TRUE FALSE
\end{verbatim}
\begin{alltt}
\hlfunctioncall{c}(\hlfunctioncall{pord_weakdom}(3:1, 5:4),  \hlfunctioncall{pord_weakdom}(5:4, 3:1))  \hlcomment{# 3:1 ?? 5:4}
\end{alltt}
\begin{verbatim}
## [1] FALSE FALSE
\end{verbatim}
\end{kframe}
\end{knitrout}


We have the following result
(Gagolewski, Grzegorzewski, \cite{GagolewskiGrzegorzewski2011:ijar}).
Let $\func{F}\in \mathcal{E}(\Ival)$. Then $\func{F}$
is symmetric, nondecreasing in each variable
and arity-monotonic if and only if for any $\vect{x},\vect{y}$
if $\vect{x}\trianglelefteq\vect{y}$, then
$\func{F}(\vect{x})\le \func{F}(\vect{y})$.
Therefore, the class of  impact functions may be equivalently
defined as all the aggregation operators
that are nondecreasing with respect to this preorder.


Additionally, we will write $\vect{x}\vartriangleleft\vect{y}$ if
$\vect{x}\trianglelefteq\vect{y}$ and $\vect{x}\neq\vect{y}$
(strict dominance).


\paragraph{Example.} Let us consider the 5 following vectors.


\begin{knitrout}\small
\definecolor{shadecolor}{rgb}{0.969, 0.969, 0.969}\color{fgcolor}\begin{kframe}
\begin{alltt}
ex1 <- \hlfunctioncall{list}(
   U = 10:0,           \hlcomment{# some upper bound}
   A = \hlfunctioncall{c}(5,5,5,5),     \hlcomment{# moderate productivity & quality}
   B = \hlfunctioncall{c}(4,3,2,1,1,0), \hlcomment{# high productivity}
   C = \hlfunctioncall{c}(8,7),         \hlcomment{# high quality}
   L = \hlfunctioncall{c}(1,1)          \hlcomment{# some lower bound}
)
\end{alltt}
\end{kframe}
\end{knitrout}


Plot of ``citation'' curves:

\begin{knitrout}\small
\definecolor{shadecolor}{rgb}{0.969, 0.969, 0.969}\color{fgcolor}\begin{kframe}
\begin{alltt}
\hlfunctioncall{for} (i in \hlfunctioncall{seq_along}(ex1))
\hlfunctioncall{plot_producer}(ex1[[i]], add=(i>1), col=i)
\hlfunctioncall{legend}(\hlstring{"topright"}, legend=\hlfunctioncall{names}(ex1), col=1:\hlfunctioncall{length}(ex1), lty=1)
\end{alltt}
\end{kframe}

{\centering \includegraphics[width=4.5in]{figures-knitr/ex1A} 

}



\end{knitrout}


get adjacency matrix for $(\{A, B, C, L, U\}, \trianglelefteq)$....

\begin{knitrout}\small
\definecolor{shadecolor}{rgb}{0.969, 0.969, 0.969}\color{fgcolor}\begin{kframe}
\begin{alltt}
ord <- \hlfunctioncall{rel_graph}(ex1, pord_weakdom)
\hlfunctioncall{print}(ord)
\end{alltt}
\begin{verbatim}
## 5 x 5 sparse Matrix of class "dtCMatrix"
##   U A B C L
## U 1 . . . .
## A 1 1 . . .
## B 1 . 1 . .
## C 1 . . 1 .
## L 1 1 1 1 1
\end{verbatim}
\begin{alltt}
\hlfunctioncall{is_reflexive}(ord)  \hlcomment{# is reflexive}
\end{alltt}
\begin{verbatim}
## [1] TRUE
\end{verbatim}
\begin{alltt}
\hlfunctioncall{is_transitive}(ord) \hlcomment{# is transitive}
\end{alltt}
\begin{verbatim}
## [1] TRUE
\end{verbatim}
\begin{alltt}
\hlfunctioncall{is_total}(ord)      \hlcomment{# not a total preorder...}
\end{alltt}
\begin{verbatim}
## [1] FALSE
\end{verbatim}
\end{kframe}
\end{knitrout}


\noindent
We see that we have $A \text{??} B$, $A \text{??} C$,
$B \text{??} C$ (no pair from $\{A,B,C\}$ is comparable
w.r.t.~$\trianglelefteq$):

\begin{knitrout}\small
\definecolor{shadecolor}{rgb}{0.969, 0.969, 0.969}\color{fgcolor}\begin{kframe}
\begin{alltt}
incomp <- \hlfunctioncall{get_incomparable_pairs}(ord)
incomp <- incomp[incomp[,1]<incomp[,2],] \hlcomment{# remove permutations: ((1,2), (2,1))->(1,2)}
incomp[,] <- \hlfunctioncall{rownames}(ord)[incomp]
\hlfunctioncall{print}(incomp) \hlcomment{# all incomparable pairs}
\end{alltt}
\begin{verbatim}
##      [,1] [,2]
## [1,] "A"  "B" 
## [2,] "A"  "C" 
## [3,] "B"  "C"
\end{verbatim}
\begin{alltt}
\hlcomment{# the other way: generate maximal independent sets}
\hlfunctioncall{lapply}(\hlfunctioncall{get_independent_sets}(ord), \hlfunctioncall{function}(set) \hlfunctioncall{rownames}(ord)[set]) 
\end{alltt}
\begin{verbatim}
## [[1]]
## [1] "A" "B" "C"
\end{verbatim}
\end{kframe}
\end{knitrout}


To draw the Hasse diagram, it will be good to de-transitivize
the graph (for \ae{}sthetic reasons)....

\begin{knitrout}\small
\definecolor{shadecolor}{rgb}{0.969, 0.969, 0.969}\color{fgcolor}\begin{kframe}
\begin{alltt}
\hlfunctioncall{require}(igraph)
hasse <- \hlfunctioncall{graph.adjacency}(\hlfunctioncall{de_transitive}(ord))
\hlfunctioncall{set.seed}(1234567) \hlcomment{# igraph's draving facilities are far from perfect}
\hlfunctioncall{plot}(hasse, layout=\hlfunctioncall{layout.fruchterman.reingold}(hasse, dim=2)) 
\end{alltt}
\end{kframe}

{\centering \includegraphics[width=4.5in]{figures-knitr/ex1C} 

}



\end{knitrout}


$(\{A, B, C, L, U\}, \trianglelefteq)$ is not totally ordered,
let's apply fair totalization (set $x\trianglelefteq'' y$
and $y\trianglelefteq'' x$ whenever $\neg (x\trianglelefteq y\text{ or }
y\trianglelefteq x)$ + calculate transitive closure


\begin{knitrout}\small
\definecolor{shadecolor}{rgb}{0.969, 0.969, 0.969}\color{fgcolor}\begin{kframe}
\begin{alltt}
ord_total <- \hlfunctioncall{closure_transitive}(\hlfunctioncall{closure_total_fair}(ord)) \hlcomment{# a total preorder}
\hlfunctioncall{print}(ord_total)
\end{alltt}
\begin{verbatim}
## 5 x 5 sparse Matrix of class "dgCMatrix"
##   U A B C L
## U 1 . . . .
## A 1 1 1 1 .
## B 1 1 1 1 .
## C 1 1 1 1 .
## L 1 1 1 1 1
\end{verbatim}
\begin{alltt}
hasse <- \hlfunctioncall{graph.adjacency}(\hlfunctioncall{de_transitive}(ord_total))
\hlfunctioncall{set.seed}(1234)
\hlfunctioncall{plot}(hasse, layout=\hlfunctioncall{layout.fruchterman.reingold}(hasse, dim=2))
\end{alltt}
\end{kframe}

{\centering \includegraphics[width=4.5in]{figures-knitr/ex1D} 

}



\end{knitrout}


...Note that each total preorder $\trianglelefteq''$ induces an
equivalence relation ($x\simeq y$ iff $x\trianglelefteq''y$ and $y\trianglelefteq''x$;
the equivalence classes may be ordered with $\trianglelefteq''$).
These may be explored with the \Rfunc{get\_equivalence\_classes()}
function....

\begin{knitrout}\small
\definecolor{shadecolor}{rgb}{0.969, 0.969, 0.969}\color{fgcolor}\begin{kframe}
\begin{alltt}
\hlfunctioncall{sapply}(\hlfunctioncall{get_equivalence_classes}(ord_total), \hlfunctioncall{function}(set) \hlfunctioncall{rownames}(ord)[set])
\end{alltt}
\begin{verbatim}
## [[1]]
## [1] "L"
## 
## [[2]]
## [1] "A" "B" "C"
## 
## [[3]]
## [1] "U"
\end{verbatim}
\end{kframe}
\end{knitrout}


Thus, we've obtained $L \prec (A \simeq B \simeq C) \prec U$.





%%%%%%%%%%%%%%%%%%%%%%%%%%%%%%%%%%%%%%%%%%%%%%%%%%%%%%%%%%%%%%%%%%%%%
%%%%%%%%%%%%%%%%%%%%%%%%%%%%%%%%%%%%%%%%%%%%%%%%%%%%%%%%%%%%%%%%%%%%%
%%%%%%%%%%%%%%%%%%%%%%%%%%%%%%%%%%%%%%%%%%%%%%%%%%%%%%%%%%%%%%%%%%%%%



\section{Aggregation Operators from the Probabilistic Perspective}

Theory of aggregation looks on the aggregation operators
from the algebraic/calculus perspective. Of course, we should
always be interested in their probabilistic properties,
e.g.~in i.i.d.~RVs models (the simplest and the most ``natural''
ones in statistics), cf.~\cite{Gagolewski2011:PhD} for discussion.

In such case we assume that input data are in fact realizations
of some random samples.

In probability, an aggregation operator
is simply called a \emph{statistic} (formalism......)

Let $(X_1,\dots,X_n)$ i.i.d.~$F$, where $\mathrm{supp}\,F = \Ival$.

....

In social phenomena modeling, if $F$ is continuous,
we often assume that the underlying density $f$ is decreasing
and convex on $\Ival$, possibly with heavy-tails.
E.g.~in the bibliometric impact assessment problem,
this assumption reflect the fact that a high paper valuation
is more difficult to obtain than the lower one,
most of the papers have very small valuation (near $0$),
and the probability of attaining a high note decreases no slower
than linearly.


\subsection{Some Notable Probability Distributions}

\subsubsection{Pareto-Type II Distribution}

Many generalizations of the Pareto distribution
have been proposed (GPD, \textit{Generalized Pareto Distributions},
cf.~e.g.~\cite{VillasenorGonzalez2009:gofgpd,Zhang2010:estgpd}).
Here we will introduce the so-called Pareto-Type II (Lomax) distribution,
which has support $\Ival=[0,\infty]$.


Formally, $X$ follows the Pareto-II distribution
with shape parameter $k>0$ and scale parameter $s>0$, 
denoted $X\sim\mathrm{P2}(k,s)$, if its density is of the form
\begin{equation}\label{Eq:Pareto2PDF}
   f(x)=\frac{k s^k}{(s+x)^{k+1}}\quad (x\ge 0).
\end{equation}
The cumulative distribution function of $X$ is then:
\begin{equation}\label{Eq:Pareto2CDF}
   F(x) = 1-\dfrac{s^k}{(s+x)^k}\quad (x\ge 0).
\end{equation}

TO DO: \package{agop}: \Rfunc{dpareto2()} -- \eqref{Eq:Pareto2PDF},
\Rfunc{ppareto2()} -- \eqref{Eq:Pareto2CDF},
and~\Rfunc{qpareto2()}... \Rfunc{rpareto2()}......

\paragraph{Properties.}
The expected value of $X\sim\mathrm{P2}(k,s)$
exists for $k>1$ and is equal to 
\[\mathbb{E}X = \frac{s}{k-1}.\]
Variance exists for $k>2$ and is equal to
\[\mathrm{Var}\, X = \frac{ks^2}{(k-2)(k-1)^2}.\]
More generally, the $i$-th raw moment for $k>i$ is given by:
\[\mathbb{E}X^i = \frac{\Gamma(i+1)\Gamma(k-i)}{\Gamma(k+1)} ks^i.\]

For a fixed $s$, if $X\sim \mathrm{P2}(k_x,s)$ and $Y\sim \mathrm{P2}(k_y,s)$,
$k_x<k_y$, then $X$ stochastically dominates $Y$,
denoted $X\succ Y$.
On the other hand, for a fixed $k$, $X\sim \mathrm{P2}(k,s_x)$, $Y\sim \mathrm{P2}(k,s_y)$, $s_x>s_y$,
implies $X \succ Y$.

Interestingly, if $X\sim\mathrm{P2}(k,s)$,
then the conditional distribution of $X-t$ given $X>t$,
is $\mathrm{P2}(k,s+t)$ $t \ge 0$.

Additionally, it might be shown that if $X\sim\mathrm{P2}(k,s)$, 
then $\ln(s+X)$ has c.d.f.~$F(x)=1-s^k e^{-kx}$ and
density $f(x)=k s^k e^{-kx}$ for $x\ge\ln s$,
i.e. has the same distribution as $Z+\ln s$, where
$Z\sim\mathrm{Exp}(k)\equiv \Gamma(1,1/k)$.




\paragraph{Parameter estimation.}
Let $\vect{x}=(x_1,\dots,x_n)$ be a realization of the Pareto-Type II
i.i.d. sample with known $s>0$.
The unbiased (corrected) maximum likelihood estimator for $k$:
% rozk³adu Pareto II rodzaju mo¿na próbowaæ wyznaczyæ metod±  
% najwiêkszej wiarogodno¶ci (MLE).
% Logarytm funkcji wiarogodno¶ci dany jest wzorem
% \begin{eqnarray}
% \ln \mathcal{L}(k,s;\vect{x}) &=& n\ln k + nk\ln s - (k+1)\sum_{i=1}^n \ln(s+x_i) \\
%                               &=& n\ln k - n\ln\tfrac{1}{s}  - (k+1)\sum_{i=1}^n \ln(1+\tfrac{1}{s} x_i).
% \end{eqnarray}
% 
% Dla znanego $s>0$ i~danej realizacji próby $(x_1,\dots,x_n)$ mamy
% \begin{equation}
% \widehat{k}=\dfrac{n}{\sum_{i=1}^n \ln\left(1+\tfrac{1}{s} x_i\right)}.
% \end{equation}
% For $n>2$ zachodzi
% \begin{eqnarray*}
% \mathbb{E}\widehat{k} &=& k \dfrac{n}{n-1},\\
% \mathrm{Var}\,\widehat{k}&=&k^2 \frac{n^2}{(n-2) (n-1)^2},
% \end{eqnarray*}
% jest to zatem  tylko asymptotycznie nieobci±¿ony estymator parametru $k$.
% £atwo jednak uwzglêdniæ poprawkê na nieobci±¿ono¶æ: \marginpar{Nieobci±¿ony estymator MLE: $\widehat{k}_{\mathrm{MLEU}}$}
\begin{equation*}
\widehat{k}(\vect{x})=\dfrac{n-1}{\sum_{i=1}^n \ln\left(1+\tfrac{1}{s} x_i\right)}.
\end{equation*}
It may be shown that for $n>2$ it holds
$\mathrm{Var}\,\widehat{k}(\vect{x})=k^2 \frac{1}{n-2}.$


TO DO: \package{agop}: 
\Rfunc{pareto2.mle\-kestimate()}

\bigskip
For both unknown $k$ and $s$ we have:
\begin{equation}\label{Eq:rownaniapochMLE}
\left\{
\begin{array}{lll}
\widehat{k} = \frac{n}{\sum_{i=1}^n \ln\left(1+x_i/\widehat{s}\right)}, \\
1+\frac{1}{n} \sum_{i=1}^n \ln\left(1+x_i/\widehat{s}\right) - \frac{n}{\sum_{i=1}^n \left(1+x_i/\widehat{s}\right)^{-1}}  &=& 0. \\
\end{array}\right.
\end{equation}
The second equation must be solved, unfortunately, numerically.
The estimation 
procedure has been implemented in \package{agop} as TO DO: \Rfunc{pareto2.mleksestimate()}....
It is worth noting that the above system of equations may sometimes
have no solution (as the local minimum of the likelihood function may not exist,
see \cite{CastilloDaoudi2009:mlegpd} for discussion).

In this case one of the estimators worth noting (and often better than MLE) was proposed
in \cite{ZhangStevens2009:estgpd}.
The Zhang-Stevens MMS (\textit{minimum mean square error}) (Bayesian)
estimator has relatively small bias (often positive) and mean squared error.
In \package{agop} it is available as TO DO: \Rfunc{pareto2.zsestimate()}.

% W~przypadku niemo¿liwo¶ci obliczenia warto¶ci estymatorów metod± najwiêkszej
% wiarogodno¶ci powinni¶my u¿yæ innego estymatora.
% Metoda momentów nie jest polecana, gdy¿ momenty zwyk³e rzêdu $i$, jak wspomniano
% wy¿ej, s± okre¶lone tylko dla $k>i$.
% Co wiêcej,  w~przypadku niektórych zaobserwowanych warto¶ci $\vect{x}$,
% równie¿ estymatory wyznaczone metod± kwantyli mog± nie byæ wyznaczalne.
% Inne, bardziej skomplikowane metody zosta³y opisane m.in.~w~pracach
% \cite{ZeaKots2001:estgpd,Luceno2006:estgpd,Zhang2010:estgpd}.
% Jak pokazuj± wyniki symulacyjne, dla nieznanego zarówno parametru
% $k$ jak i~$s$, estymatory te cechuj± siê lepszymi w³asno¶ciami ni¿ estymator
% najwiêkszej wiarogodno¶ci.
% 
% \bigskip\label{Ozn:MMSE}
% Warto zwróciæ szczególn± uwagê na metodê szacowania warto¶ci parametrów zaproponowan±
% przez Zhanga i Stevensa \cite{ZhangStevens2009:estgpd}.
% Przedstawiany estymator
% cechuje siê niewielkim obci±¿eniem (najczê¶ciej
% dodatnim), wzglêdnie ma³± wariancj± oraz tym, ¿e da siê
% wyznaczyæ jego warto¶æ w~prawie wszystkich przypadkach.



% Niech $\theta := -\frac{1}{s}$, gdzie                                           
% $-1\le \theta < 0$.
% Poszukujemy estymatora bayesowskiego minimalizuj±cego ¶redniokwadratow±
% funkcjê ryzyka
% (estymator MMSE, ang. \textit{minimum mean square error estimator}),
% który jest warto¶ci± oczekiwan± rozk³adu a~posteriori
% \begin{equation}\label{Eq:MMSEbayes}
% \widehat{\theta} = \Expectation(\theta | \vect{X}=\vect{x}) = \int_{-1}^0 \theta\, \pi(\theta|\vect{x})\,d\theta,
% \end{equation}
% gdzie
% \begin{equation}
% \pi(\theta|\vect{x})=\frac{h(\theta; \vect{x}) \exp[{\ell(\theta; \vect{x})}]}{\int_{-1}^0 h(\theta; \vect{x}) \exp[{\ell(\theta; \vect{x})}]  \,d\theta},
% \end{equation}
% dla  pewnej gêsto¶ci rozk³adu
% a~priori $h(\theta; \vect{x})$ oraz
% logarytmu z~profilowej funkcji wiarogodno¶ci (ang. \textit{profile log-likelihood})
% parametru $\theta$ postaci
% \[\ell(\theta; \vect{x}):=n\left( \ln(-{k}^*\theta)-1-1/{k}^* \right),\]
% gdzie
% \[{k}^* = n \left[\sum_{i=1}^n \ln\left(1-\theta x_i\right) \right]^{-1}.\]
% 
% W~rozpatrywanej pracy stosowane jest nastêpuj±ce
% przybli¿enie numeryczne wzoru \eqref{Eq:MMSEbayes}.
% Dla $m=20+\lfloor n\rfloor$ i~$j=1,\dots,m$ niech
% \[
% \vartheta_j = 1/x_{(n)} + \left[ 1-\sqrt{\frac{m}{j-0{,}5}} \right] \textfractionsolidus (3\, x_{(\lfloor n/4+0.5\rfloor)}).
% \]
% \clearpage\noindent
% Ponadto, niech $w(\vartheta_j) = \exp(\ell(\vartheta_j; \vect{x}))/\sum_{i=1}^{m} \exp(\ell(\vartheta_i;\vect{x}))$.
% W~konsekwencji, estymatorem parametru skali bêdzie
% $
%    \widehat{s}_\text{ZS} = -1/\sum_{j=1}^m \vartheta_j w(\vartheta_j).
% $
% Warto¶æ $\widehat{k}_\text{ZS}$ wyznaczamy z~pierwszego równania                \marginpar{Estymatory MMSE: $\widehat{k}_\text{ZS}$~i~$\widehat{s}_\text{ZS}$}
% uk³adu \eqref{Eq:rownaniapochMLE}.
% Jako rozk³adu a~priori $h(\theta; \vect{x})$ u¿ywamy tutaj
% gêsto¶ci rozk³adu $\mathrm{P2}(2, 1/3\, x_{(\lfloor n/4+0.5\rfloor)})$
% (por.~motywacjê i~dyskusjê w~\cite{ZhangStevens2009:estgpd}).
% Algorytm wyznaczaj±cy estymatory przedstawion± metod±
% zosta³ zaimplementowany w~bibliotece \CITAN{} w~funkcji \Rfunc{pareto2.zsestimate()},
% zob.~rozdz.~\ref{Func:pareto2.zsestimate}.
% 

\paragraph{Goodness-of-fit tests.}
TO BE DONE....
% \paragraph{Testowanie zgodno¶ci.}\label{Ozn:P2GoF}
% Przedstawione estymatory mog± byæ u¿yte                                         \marginpar{GoF}
% w~procedurze  testowania zgodno¶ci %(ang. \textit{goodness-of-fit tests})
% danej zaobserwowanej próbki losowej z~rozk³adem Pareto II rodzaju.
% W~tym celu mo¿na u¿yæ np. testu Ko³mogorowa,
% Cram\'{e}ra-von Misesa b±dŒ Andersona-Darlinga (który przyk³ada wiêksz± wagê
% od pozosta³ych do obserwacji w~ogonach rozk³adu),
% por.~analizê w³asno¶ci tych testów w~pracach \cite{ChoulakianStephens2001:gofgpd,
% ZhangStevens2009:estgpd}.
% 
% W~bibliotece \CITAN{} udostêpnili¶my funkcjê funkcjê \Rfunc{pareto2.gof\-test()}
% (zob. rozdz.~\ref{Func:pareto2.goftest}),
% która domy¶lnie stosuje test Andersona-Darlinga,
% w~którym rozk³ad statystyki przybli¿ony jest za pomoc± metody zaproponowanej
% w~\cite{MarsagliaMarsaglia2004:adtest}.
% To postêpowanie (wraz z omówion± wy¿ej metod± estymacji) jest równowa¿ne
% z~przedstawionym w~pracy \cite{ZhangStevens2009:estgpd}.
% 
% 
% 


\paragraph{Applications.}
TO DO
% 
% Rozk³ad Pareto II rodzaju stosuje siê czêsto do modelowania 
%  tzw.~warto¶ci ekstremalnych [por.~\citealp{ChoulakianStephens2001:gofgpd}].
% Jest on zalecany w~literaturze bibliometrycznej m.in. do opisu
% rozk³adu liczby cytowañ, por. np. \cite{Glanzel2008:hconcatenation}
% % (tutaj wystêpuje sugestia, ¿e mo¿na przyj±æ $k\simeq 2$)
% oraz prace \cite{Glanzel2007:hnewapplications,BarczaTelcs2009:paretoimplyh,
% Glanzel2008:statbibh}. Czêsto zak³ada siê ponadto, i¿ $s\ge 1$.
% 



\paragraph{Two-sample $F$-test.}
The following simple test was introduced in \cite{Gagolewski2011:PhD}.
Let
$(X_1,X_2,\dots,X_{n_1})$ $i.i.d.$ $\mathrm{P2}(k_1,s)$ and
$(Y_1,Y_2,\dots,Y_{n_2})$ $i.i.d.$ $\mathrm{P2}(k_2,s)$,
where $s$ is an a-priori known scale parameter.
We are going to verify the null hypothesis
$H_0: k_1=k_2$ against the two-sided alternative hypothesis
$K: k_1\neq k_2$.


% Mo¿na pokazaæ, ¿e dla $X_i\sim\mathrm{P2}(k,s)$, $\ln(s+X_i)$
% ma rozk³ad o~dystrybuancie
% $F(x)=1-s^k e^{-kx}$ i gêsto¶ci $f(x)=k s^k e^{-kx}$ dla $x\ge\ln s$,
% czyli ma taki sam rozk³ad jak zmienna losowa $Z+\ln s$, gdzie
% $Z\sim\mathrm{Exp}(k)\equiv \Gamma(1,1/k)$.
% Zatem $\sum_{i=1}^n \ln(s+X_i)-n\ln s\sim\Gamma(n,1/k)$.
It might be shown that  $\sum_{i=1}^n \ln(s+X_i)-n\ln s\sim\Gamma(n,1/k)$.
This implies that under $H_0$, the following test statistic
follows the Snedecor $\mathrm{F}$ distribution:
% \begin{equation}
% T(\vect{X},\vect{Y})=\dfrac{n_1}{n_2}
%    \dfrac{\sum_{i=1}^{n_2} \ln\left(s+Y_i\right)-n_2\ln s}
%          {\sum_{i=1}^{n_1} \ln\left(s+X_i\right)-n_1\ln s}
% \stackrel{H_0}{\sim}\mathrm{F}^{[2n_2,2n_1]}.
% \end{equation}
\begin{equation}
R(\vect{X},\vect{Y})=\dfrac{n_1}{n_2}
   \dfrac{\sum_{i=1}^{n_2} \ln\left(1+\frac{Y_i}{s}\right)}
         {\sum_{i=1}^{n_1} \ln\left(1+\frac{X_i}{s}\right)}
\stackrel{H_0}{\sim}\mathrm{F}^{[2n_2,2n_1]}.
\end{equation}
The null hypothesis is accepted iff
\[
R(\vect{x},\vect{y})\in\left[\text{\Rfunc{qf}}(\tfrac{\alpha}{2},2n_2,2n_1),\,
\text{\Rfunc{qf}}(1-\tfrac{\alpha}{2},2n_2,2n_1)\right],
\]
where $\text{\Rfunc{qf}}(q,d_1,d_2)$ denotes the $q$-quantile
of $\mathrm{F}^{[d_1,d_2]}$

The $p$-value may be determined as follows:
\begin{equation}
p=2\left(\tfrac{1}{2}-\left|\text{\Rfunc{pf}}(R(\vect{x},\vect{y}), 2n_2, 2n_1)-\tfrac{1}{2}\right|\right),
\end{equation}
where $\text{\Rfunc{pf}}(x,d_1,d_2)$ 
is the c.d.f.~of $\mathrm{F}^{[d_1,d_2]}$.

\medskip
TO DO: \Rfunc{pareto2.\allowbreak ftest()}.




\subsection{Stochastic Properties of Aggregation Operators}

OWA, L-statistics

OWMax, S-statistics

$h$-index and its distribution









%%%%%%%%%%%%%%%%%%%%%%%%%%%%%%%%%%%%%%%%%%%%%%%%%%%%%%%%%%%%%%%%%%%%%
%%%%%%%%%%%%%%%%%%%%%%%%%%%%%%%%%%%%%%%%%%%%%%%%%%%%%%%%%%%%%%%%%%%%%
%%%%%%%%%%%%%%%%%%%%%%%%%%%%%%%%%%%%%%%%%%%%%%%%%%%%%%%%%%%%%%%%%%%%%



\section{NEWS/CHANGELOG}

\begin{knitrout}\small
\definecolor{shadecolor}{rgb}{0.969, 0.969, 0.969}\color{fgcolor}\begin{kframe}
\begin{verbatim}
                          ** agop package NEWS **

***************************************************************************

0.1-0 /under development/

* initial release
  [the package started as a lightweight fork of the CITAN package]

***************************************************************************
\end{verbatim}
\end{kframe}
\end{knitrout}



\paragraph{Acknowledgments.}
This document has been generated with \LaTeX, \package{knitr} and
the \package{tikzDevice} package for \R.
Their authors' wonderful work is fully appreciated.


The contribution of Marek Gagolewski was partially supported
by the European Union from resources of the European Social Fund, Project PO KL 
``Information technologies: Research and their interdisciplinary
applications'', agreement UDA-POKL.04.01.01-00-051/10-00 (March-June 2013),
and by FNP START Scholarship from the Foundation for Polish Science (2013).



\bibliographystyle{acm}
\bibliography{bibliography}

\end{document}
